\documentclass[a4paper, 12pt]{article}
\usepackage{comment}
\usepackage{lipsum}
\usepackage{fullpage}
\usepackage[a4paper, total={7in, 10in}]{geometry}
\usepackage{amsmath}
\usepackage[utf8]{inputenc}
\usepackage[russian]{babel}
\usepackage{amssymb,amsthm}

\newtheorem{theorem}{Theorem}
\newtheorem{corollary}{Corollary}
\usepackage{graphicx}
\usepackage{tikz}
\usetikzlibrary{arrows}
\usepackage{verbatim}
\usepackage{xcolor}
\usepackage{mdframed}
\usepackage[shortlabels]{enumitem}
\usepackage{indentfirst}
\setlength{\parindent}{0cm}
\usepackage{hyperref}
\usepackage{float}

\usepackage{setspace}
\setlength{\parindent}{20pt}
\setlength{\parskip}{4pt}

\graphicspath{{./images/}}

\newcommand{\beq}{\begin{equation}}
\newcommand{\eeq}{\end{equation}}

\begin{document}

\setstretch{1.1}

\section*{Задача 2. Социальное моделирование.}

К 1 сентября в стране <<Гаудэамус игитур>> $m$ абитуриентов $A=\{a_1,a_2,...,a_m\}$ пытаются поступить в $n$ университетов $C=\{c_1, c_2, ..., c_n\}$.
Поступление в университет происходит по результатам трех выпускных экзаменов <<АГУ>>, за каждый из которых абитуриент может получить до 100 баллов.
Пусть у каждого абитуриента заданы строгие предпочтения на множестве университетов в виде списка $p(a_j)=\{c_{j1},  c_{j2}, ..., c_{jk}\}$, при этом некоторые университеты могут отсутствовать в списках предпочтений отдельных абитуриентов.
Каждый университет имеет нижнюю квоту поступления $l(c_i)$ и верхнюю квоту поступления $u(c_i)\geqslant l(c_i)$ для любого $c_i\in C$.
В случае, если число абитуриентов, поступивших в университет, меньше нижней квоты, то университет подлежит закрытию.
Будем говорить, что имеется стабильное разбиение абитуриентов по университетам, если выполнены следующие условия:

a) не существует открытого (работающего) университета, такого, что существует абитуриент, который наряду с открытым университетом предпочитают друг друга своим текущим
<<партнерам>>;

b) не существует закрытого (не работающего) университета, такого, что существует абитуриент, который предпочитает закрытый университет своему текущему университету.

\textbf{Задания.}

1) Предположим, что предпочтения всех абитуриентов, которые хотят стать историками, по университетам совпадают.
Валя и Маша являются абитуриентами и тоже хотят стать историками.
Перед тем как начнется зачисление в университеты, Маша думает рассказать ли Вале про новый <<Перспективный университет>>, о котором Валя не знала.
Может ли получиться, что:

(a) (2 балла) Маша окажется в менее предпочтительном университете, рассказав Вале о новом университете?

(b) (2 балла) Маше выгодно рассказать Вале о новом университете?
\\

2) (5 баллов) Друзья Миша, Саша, Петр и Валера живут в Простоквашино и хотят в следующем году поступать в университет.
Ребята заочно знакомы с девушками Дарьей и Натальей из Уфы, Катей и Лизой из Москвы, и у каждого юноши и девушки есть свой список предпочтений противоположного пола.
Поэтому Миша, Саша, Петр и Валера хотели бы поступать в университеты тех городов, где у них и у девушек возникнут обоюдные симпатии.
Единственная проблема, что братья Миша и Саша не хотят расставаться, и поедут обязательно вместе или в Уфу, или в Москву.
Ребятам настолько важно быть вместе, что любой из них готов отказаться от самой предпочтительной девушки, если она будет жить не в том городе, где живет девушка брата.
Возможно ли, при любых вариантах предпочтений найти стабильное разбиение на пары?
\\

3) (6 баллов) Исследовать условия, при которых можно распределить $m$ абитуриентов $A=\{a_1, a_2, ..., a_m\}$ по $n$ университетам $C=\{c_1, c_2, ..., c_n\}$ так, чтобы итоговое разбиение было устойчивым.
Всегда ли это возможно?
\\

4) (15 баллов) а) Составьте распределение абитуриентов по университетам, если известны баллы абитуриента, его список предпочтений, а также верхняя квота университетов (нижняя квота для всех университетов равна нулю).
Если баллы абитуриентов совпадают, то лучший абитуриент определяется по лексикографическому порядку его id номера.
В результате должен быть получен файл, содержащий список студентов и университетов, с указанием куда поступил абитуриент.\\

data.csv -- файл, содержащий предпочтения студентов (id\_s000...am) относительно университетов (id\_u00...bn), а также личный рейтинг студентов в баллах в столбце rating\_students;

quota.json -- файл, содержащий верхние квоты университетов;

example.json -- пример файла-ответа.\\

б) Возможно ли использование данного алгоритма в текущей приемной кампании для оптимального зачисления абитуриентов в университеты.
Какие положительные и отрицательные стороны, в том числе социальные, может иметь алгоритм при использовании в России?
\\

5) (15 баллов) Докажите, хотя бы в частных случаях, например, в случае верхней квоты равной 3, что задача стабильного разбиения абитуриентов по университетам с нижней и верхней ненулевыми квотами в общей постановке является NP-полной?
\\

Для доказательства можно воспользоваться следующим утверждением.
Рассмотрим множество мужчин $V=\{m_1, m_2, ..., m_n\}$ и женщин $U=\{w_1, w_2, ..., w_n\}$.
У каждой женщины и мужчины есть свой список предпочтений, возможно, не полный и нестрого упорядоченный, а именно этот список может содержать неупорядоченную подпоследовательность предпочтений, например $w_i: m_1, (m_2,m_3), m_4, m_5$ (в этом случае для женщины $w_i$ нет явного предпочтения между мужчинами $m_2$, $m_3$).
Требуется решить задачу о стабильном образовании пар мужчин и женщин.
Справедливо утверждение, что данная задача является NP-полной.
\\

Исходные данные представлены в директории задачи.

В пункт 2.1. <<Загрузка файла решения>> необходимо загрузить файл в формате pdf/jpeg.
Файл должен содержать полное решение задачи 2.

В пункт 2.2 <<Загрузка файла решения json>> необходимо загрузить файл в формате json.
Файл должен содержать решение задачи 2 пункт 4.
Пример файла ответа example.json.

\end{document}
