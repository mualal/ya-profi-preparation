\documentclass[a4paper, 12pt]{article}
\usepackage{comment}
\usepackage{lipsum}
\usepackage{fullpage}
\usepackage[a4paper, total={7in, 10in}]{geometry}
\usepackage{amsmath}
\usepackage[utf8]{inputenc}
\usepackage[russian]{babel}
\usepackage{amssymb,amsthm}

\newtheorem{theorem}{Theorem}
\newtheorem{corollary}{Corollary}
\usepackage{graphicx}
\usepackage{tikz}
\usetikzlibrary{arrows}
\usepackage{verbatim}
\usepackage{xcolor}
\usepackage{mdframed}
\usepackage[shortlabels]{enumitem}
\usepackage{indentfirst}
\setlength{\parindent}{0cm}
\usepackage{hyperref}
\usepackage{float}

\usepackage{setspace}
\setlength{\parindent}{20pt}
\setlength{\parskip}{4pt}

\graphicspath{{./images/}}

\newcommand{\beq}{\begin{equation}}
\newcommand{\eeq}{\end{equation}}

\begin{document}

\setstretch{1.1}

\section{Задача 1. Моделирование роста гидратного слоя.}

Газовые гидраты -- твёрдые кристаллические соединения, которые образуются при контакте воды и газа в определённых термобарических условиях.
Молекулы воды в гидрате образуют сложный трёхмерный каркас, называемый клатратом, в полостях которой размещаются молекулы газа.
Клатратной структурой гидрата объясняется его удивительное свойство -- в одном объёме гидрата может содержаться до 180 объёмов газа.

Борьба с отложением гидратов в стволах газовых и нефтяных скважин, на стенках промысловых и магистральных трубопроводов является одной из актуальных проблем при добыче и транспортировке нефти и газа.
Образующиеся на стенках труб гидратные отложения резко уменьшают их пропускную способность, могут привести к их закупорке, возникновению аварийных ситуаций и даже к взрывам.

Одной из неожиданных сфер применения газовых гидратов является борьба с парниковым эффектом -- на протяжении ряда лет учёные изучают возможность захоронения углекислого газа атмосферы в газогидратной форме в подземных коллекторах криолитозоны.\\

\textbf{Задания:}

1) Построить математическую модель роста гидратного слоя и обосновать её адекватность с учётом следующих допущений.
\begin{itemize}
	\item В начальный момент времени углекислый газ и вода занимают полупространства $x<0$ и $x>0$, соответственно.
	На границе углекислого газа и воды образуется и начинает расти слой газогидрата.
	Плоская граница контакта между водой и газогидратом (граница гидратообразования) движется вправо, а плоская граница контакта между углекислым газом и газогидратом неподвижна.
	\item В слое газогидрата присутствуют неподвижный углекислый газ, находящийся в составе гидрата (связанный с водой), и подвижный (диффундирующий) углекислый газ.
	Скорость процесса образования газогидрата лимитируется только процессом диффузионного переноса подвижного углекислого газа в газогидрате (диффузией воды можно пренебречь).
	\item Подвижный углекислый газ мгновенно переходит в состав гидрата при достижении границы контакта с водой.
	\item Плотность подвижного углекислого газа в газогидрате на границе контакта <<газогидрат-газ>> в $\mu$ раз меньше плотности углекислого газа в области $x<0$.
\end{itemize} 

2) Получить аналитическое решение для частного случая, когда скорость изменения плотности подвижного углекислого газа в любой заданной точке слоя газогидрата пренебрежимо мала.

3) Выбрать и обосновать численный метод решения уравнений математической модели в полной постановке (без учёта допущения из пункта 2).

4) Определить толщину слоя газового гидрата через 90, 180 и 360 суток после начала процесса гидратообразования (ответ дать в миллиметрах, округлить до второго знака после запятой).

Использовать следующие значения параметров: относительное массовое содержание углекислого газа в составе газогидрата $G=0.28$; коэффициент диффузии газа в газогидрате $D_g=10^{-12}\text{ м}^2/\text{с}$; плотность газогидрата $\rho_h=1100\text{ кг}/\text{м}^3$; давление углексилого газа ($x<0$) $p=6\text{ МПа}$; температура углекислого газа ($x<0$) $T=275\text{ К}$; $\mu=10$.\\

Максимальный балл в случае успешного решения задачи 1 -- 45 баллов.
В случае частичного выполнения задачи критерии оценивания по задаче 1 следующие:

1) оценка решения участника -- 25 баллов (максимальный балл), из них

\hspace{7pt} а) за представление корректной математической модели роста гидратного слоя -- 10 баллов;

\hspace{10pt} б) за получение аналитического решения для частного случая, когда скорость изменения плотности подвижного газа в любой заданной точке слоя газогидрата пренебрежимо мала -- 5 баллов;

\hspace{10pt} в) за выбор и применение численного метода решения уравнений математической модели в полной постановке -- 5 баллов;

\hspace{10pt} г) за расчёт координат границы гидратообразования в заданные моменты времени -- 5 баллов;

2) оценка выступления участника -- 20 баллов (максимальный балл), из них

\hspace{10pt} а) за обоснование математической модели роста гидратного слоя и её адекватности -- 10 баллов;

\hspace{10pt} б) за обоснование аналитического решения для частного случая -- 5 баллов;

\hspace{10pt} в) за обоснование метода решения уравнений математической модели в полной постановке -- 5 баллов.

Отчёт должен быть загружен в формате pdf/jpeg.

Обратите внимание, что вы можете загрузить только один файл.
Размер файла не должен превышать 10MB.


\end{document}
